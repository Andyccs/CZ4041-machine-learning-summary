\chapter{Decision Tree}

\begin{multicols}{2}
\section{Hunt's Algorithm}

\noindent Hunt's algorithm grows a decision tree in a recursive fashion by partitioning the training records into successively purer subsets. Let $D_t$ be the set of training records that reach a node $t$:

\begin{itemize}
    \item If $D_t$ contains records that belong the same class $y_t$, then $t$ is a leaf node labeled as $y_t$
    \item If $D_t$ is an empty set, then t is a leaf node labeled by the default class
    \item If $D_t$ contains records that belong to more than one class, use an \textit{attribute test} to split the data into smaller subsets.
\end{itemize}

\noindent We use greedy strategy to split the data based on an attribute test that optimizes certain criterion. 

\section{Determine the Best Split}

\noindent Greedy approach: Nodes with homogeneous class distribution are preferred

\noindent Measure of node impurity, Entrophy:

$$Entrophy(t)=-\sum P(j \mid t) log_2 P(j \mid t)$$ 

\noindent Maximum value: $log_{2} K$, where $K$ is the total number of all possible values of Y. This happens when records are equally distributed among all classes, implying least information. 

\noindent Minimum value: 0, when all records belong to one class, implying most information. \\

\noindent The information gain by splitting a parent node into child nodes are:

$$GAIN_{split}=Entrophy(p)-\sum_{i=1}^{k} \frac{n_i}{n}Entrophy(i)$$
\noindent We choose the split that give us the most information gain. Disadvantage: Tends to prefer splits that result in large number of partitions, each being small but pure. \\

\noindent Introduce Gain Ratio: 

$$GainRATIO_{split}=\frac{GAIN_{split}}{SplitINFO}$$

$$SplitINFO=-\sum_{i=1}^{k} \frac{n_i}{n}log_2 \frac{n_i}{n}$$ \\

\noindent In gain ratio, we adjust information gain by the entropy of partitioning (SplitINFO). Higher entropy partitionaing (large number of small partitions is penalized). 

\section{Stopping Criteria for Tree Induction}

Stop expanding a node when:
\begin{itemize}
    \item all the records belong to the same class
    \item all the recrods have similar attribute values
    \item use early termination
\end{itemize}

\end{multicols}
