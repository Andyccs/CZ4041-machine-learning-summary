\chapter{Dimensionality Reduction}

\begin{multicols}{2}
\noindent We reduce dimension:
\begin{enumerate}
    \item To reduce time and space complexity
    \item For data visualisation 
\end{enumerate}

\noindent Feature selection means we choose $k<d$ important features, by using subset selection algorithms. \\

\noindent Feature extraction means we project the original $d$ dimensions to new $k<d$ dimensions.

\section{Math: Eigenvalues and Eigenvectors}

\noindent Let $\mathbf{A}$ be a square matrix and $\lambda$ be an eigenvalue. If $\mathbf{x}$ is an eigenvector of $\mathbf{A}$, then:
$$\mathbf{Ax} = \lambda \mathbf{x}$$ 
$$(\mathbf{A}-\lambda \mathbf{I})\mathbf{x} = 0$$

\noindent Since we don't want $\mathbf{x}$ to be 0, then $|\mathbf{A} - \lambda \mathbf{I}|=0$, because if the determinant is not equal to zero, then $\mathbf{x} = 0$:

$$(\mathbf{A}-\lambda \mathbf{I})^{-1}(\mathbf{A}-\lambda \mathbf{I})\mathbf{x} = (\mathbf{A}-\lambda \mathbf{I})^{-1} \times 0$$
$$\mathbf{x} = 0$$

\section{Principal Components Analysis}

\noindent Find a low-dimensional space such that when $\mathbf{x}$ is projected there, information loss is minimized. The project of $\mathbf{x}$ on the direction of $\mathbf{w}$ is:
$$z=\mathbf{w}^{T}\mathbf{x}=|w||x| \cos \theta$$

\noindent Goal: find $\mathbf{w}$ such that $Var(z)$ is maximized
\begin{equation*}
\begin{split}
    Var(z) &= Var(\mathbf{w}^{T}\mathbf{x}) \\
    &= \mathbf{w}^{T} E[(\mathbf{x}-\mathbf{\mu})(\mathbf{x}-\mathbf{\mu})^{T}] \mathbf{w} \\
    &= \mathbf{w}^{T} \Sigma \mathbf{w}\\
    Var(\mathbf{x}) &= \Sigma \\
    &= \frac{1}{m} \sum_{i=1}^n \mathbf{x}^{(i)} (\mathbf{x}^{(i)})^{T}
\end{split}
\end{equation*}

\noindent After maximizing $Var(z)$, we will get $\mathbf{W}$, which the columns are the eigenvectors of $\Sigma$, with dimension $d\times d$. The reduced data $z$:
$$\mathbf{z} = \mathbf{W}_{\text{reduce}}^{T} (\mathbf{x} - \mathbf{\mu})$$

\noindent where $\mathbf{W}_{\text{reduce}}$ has a dimension of $(d\times k), k<d$, and $\mathbf{z}$ has a dimension of $(k\times 1)$. To reconstruct the approximate data $\mathbf{x}$:
$$x_{\text{approx}} = \mathbf{W}_{\text{reduce}} \mathbf{z}$$

\subsection{Choosing the Dimension}

\noindent The average squared projection error is:
$$A = \frac{1}{m} \sum_{i=1}^m \| x^{(i)} - x^{(i)}_{\text{approx}}\|^2$$

\noindent Total variation in the data:
$$B = \frac{1}{m} \sum_{i=1}^m \| x^{(i)}\|^2$$

\noindent Typically, we choose $k$ to be smallest value so that 90\% of variance is retained:
$$\frac{A}{B} \le 0.1$$

\noindent It can be proved that the above equation is equivalent to:
$$1 - \frac{\lambda_1+\lambda_2+\ldots+\lambda_k}{\lambda_1+\lambda_2+\ldots+\lambda_k+\ldots+\lambda_d} \le 0.1$$
$$\frac{\lambda_1+\lambda_2+\ldots+\lambda_k}{\lambda_1+\lambda_2+\ldots+\lambda_k+\ldots+\lambda_d} > 0.9$$

\section{Math: Spectral Decomposition}
\noindent Spectral decomposition is the factorization of a matrix into a canonical form. Let $\mathbf{S}$ be a square $(N\times N)$ matrix, then it can be factorized as:
$$\mathbf{S}=\mathbf{WDW}^{T}$$

\noindent where $\mathbf{W}$ is the square $(N\times N)$ matrix whose ith column is the eigenvector $w_i$ of $\mathbf{S}$, and $\mathbf{D}$ is the diagonal matrix whose diagonal elements are the corresponding eigenvalues.

\section{Feature Embedding}
\noindent When $\mathbf{X}$ is a $N\times d$ data matrix, the covariance matrix is $\mathbf{X}^{T} \mathbf{X}$ with dimension $d\times d$ (if mean-centered). This is use by PCA. \\

\noindent The pairwise similarities of instance is $\mathbf{X} \mathbf{X}^{T}$ with $N\times N$ dimension. This is used in faeture embedding:
$$\mathbf{X} \mathbf{X}^{T} = \mathbf{VEV}^{T}$$

\noindent where $\mathbf{V}$ has eigenvectors of $\mathbf{X} \mathbf{X}^{T}$ as column with dimension $N\times N$, $\mathbf{E}$ is a diagonal $N\times N$ matrix with eigenvalues. After factorizing, we take $V_{\text{reduce}}$ with dimension $N\times k$, where $k<N$, as features. Feature embedding is preferred when $N<d$. 

\section{Math: Singular Value Decomposition}
\noindent Singular value decomposition:
$$\mathbf{X} = \mathbf{VAW}^{T}$$

\noindent $\mathbf{V}$ is $N\times N$ and contains the eigenvectors of $\mathbf{XX}^T$
\noindent $\mathbf{W}$ is $d\times d$ and contains the eigenvectors of $\mathbf{X}^T \mathbf{X}$
\noindent $\mathbf{A}$ is $N\times d$ contains singular values on its first $k$ diagonal

\end{multicols}
